\subsection{Statics}

\subsubsection{Expressions}

A channel context $\Sigma = \Has{a_1}{\tau_1}, \dots, \Has{a_n}{\tau_n}$ describes the names and associated types of channels that are available for use.

\setcounter{infercount}{1}
\renewcommand{\infern}[2]{\inferrule{#1}{#2}(\text{SE}_{\arabic{infercount}}\stepcounter{infercount})}
\fbox{$\jse{\Gamma}{\Sigma}{e}{\tau}$}
\begin{mathpar}
  \infern
  {\jsm{\Gamma}{\Sigma}{m}{\tau}}
  {\jse{\Gamma}{\Sigma}{\tcmdabt{\tau}{m}}{\gcmdtyabt{\tau}}}

  \infern
  { }
  {\jse{\Gamma}{\Sigma, \Has{a}{\tau}}{\chabt{}{a}}{\chtyabt{\tau}}}
\end{mathpar}

\subsubsection{Commands}

\setcounter{infercount}{1}
\renewcommand{\infern}[2]{\inferrule{#1}{#2}(\text{SM}_{\arabic{infercount}}\stepcounter{infercount})}
\fbox{$\jsm{\Gamma}{\Sigma}{m}{\tau}$}
\begin{mathpar}
  \infern
  {\jse{\Gamma}{\Sigma}{e}{\tau}}
  {\jsm{\Gamma}{\Sigma}{\retcmdabt{e}}{\tau}}

  \infern
  {\jse{\Gamma}{\Sigma}{e}{\gcmdtyabt{\tau}} \\
    \jsm{\Gamma,x:\tau}{\Sigma}{m}{\tau'}
  }
  {\jsm{\Gamma}{\Sigma}{\seqcmdabt{e}{x}{m}}{\tau'}}

  \infern
  {\jse{\Gamma}{\Sigma}{e}{\gcmdtyabt{\tau}}
  }
  {\jsm{\Gamma}{\Sigma}{\spawncmdabt{e}}{\chtyabt{\tau}}}

  \infern
  {\jse{\Gamma}{\Sigma}{e_1}{\chtyabt{\tau}} \\
    \jse{\Gamma}{\Sigma}{e_2}{\tau}
  }
  {\jsm{\Gamma}{\Sigma}{\ssendrefcmdabt{e_1}{e_2}}{\unittyabt}}

  \infern
  {\jse{\Gamma}{\Sigma}{e}{\chtyabt{\tau}}}
  {\jsm{\Gamma}{\Sigma}{\synccmdabt{e}}{\tau}}

  \infern
  {\jsm{\Gamma}{\Sigma, \Has{a}{\tau}}{m}{\tau'}}
  {\jsm{\Gamma}{\Sigma}{\newchncmdabt{a}{\tau}{m}}{\tau'}}
\end{mathpar}

\subsubsection{Processes}

\paragraph{Atomic Processes}
Atomic processes $\atproccst{a}{m}$ contain commands $m$, which will be user-written programs.
It will be arranged in the dynamics that the channel $a$ of an atomic process eventually receives the result of the command $m$.

\begin{remark}
  Processes and actions are not user-level constructs, like stacks $k$ and states $s$ in \LangKPCF{}.
  Programmers will write commands $m$, \emph{not} processes $p$.
\end{remark}


\subsubsection{Process Typing}

The judgment $\jsp{\Sigma}{p}$ describes the valid processes relative to the signature $\Sigma$ and context $\Gamma$. Processes have to be typed under a specific context due to the existence of accepting processes.

\setcounter{infercount}{1}
\renewcommand{\infern}[2]{\inferrule{#1}{#2}(\text{SP}_{\arabic{infercount}}\stepcounter{infercount})}
\fbox{$\jsp{\Sigma}{p}$}
\begin{mathpar}
  \infern
  { }
  {\jsp{\Sigma}{\stopproccst}}

  \infern
  {\jsp{\Sigma}{p_1} \\ \jsp{\Sigma}{p_2}}
  {\jsp{\Sigma}{\concproccst{p_1}{p_2}}}

  \infern
  {\jsp{\Sigma, \Has{a}{\tau}}{p}}
  {\jsp{\Sigma}{\tnewproccst{\tau}{a}{p}}}

  \infern
  {\jsm{\Gamma}{\Sigma, \Has{a}{\tau}}{m}{\tau}}
  {\jsp{\Sigma, \Has{a}{\tau}}{\atproccst{a}{m}}}

  \infern
  {\jse{\Gamma}{\Sigma, \Has{a}{\tau}}{e}{\tau}}
  {\jsp{\Sigma, \Has{a}{\tau}}{\sendproccst{a}{e}}}

  \infern
  {\jspc{\Gamma, x : \tau}{\Sigma, \Has{a}{\tau}}{p} }
  {\jsp{\Sigma, \Has{a}{\tau}}{\recvproccst{a}{x}{p}} }
\end{mathpar}

\subsubsection{Process Structural Congruence}

Processes $p$ are identified up to structural congruence, an equivalence relation written $\jspeq{p_1}{p_2}$.

\setcounter{infercount}{1}
\renewcommand{\infern}[2]{\inferrule{#1}{#2}(\text{E}_{\arabic{infercount}}\stepcounter{infercount})}
\fbox{$\jspeq{p_1}{p_2}$}

First, we state that $\jspeq{p_1}{p_2}$ is an equivalence relation (reflexive, symmetric, and transitive):
\begin{mathpar}
  \infern
  {\AlphaEq{p_1}{p_2}}
  {\jspeq{p_1}{p_2}}
  \and
  \infern
  {\jspeq{p_2}{p_1}}
  {\jspeq{p_1}{p_2}}
  \and
  \infern
  {\jspeq{p_1}{p_2} \\ \jspeq{p_2}{p_3}}
  {\jspeq{p_1}{p_3}}
\end{mathpar}
Then, we state that $\jspeq{p_1}{p_2}$ is a congruence:
\begin{mathpar}
  \infern
  {\jspeq{p_1}{p_1'} \\ \jspeq{p_2}{p_2'}}
  {\jspeq{\concproccst{p_1}{p_2}}{\concproccst{p_1'}{p_2'}}}
  \and
  \infern
  {\jspeq{p}{p'}}
  {\jspeq{\tnewproccst{\tau}{a}{p}}{\tnewproccst{\tau}{a}{p'}}}
  \and
  \infern
  {\jspeq{p}{p'}}
  {\jspeq{\recvproccst{a}{x}{p}}{\recvproccst{a}{x}{p'}}}
\end{mathpar}

We guarantee that $\stopproccst$ and $\concproccst{-}{-}$ form a commutative monoid:
\begin{mathpar}
  \infern
  {\strut}
  {\jspeq{\concproccst{p_1}{(\concproccst{p_2}{p_3})}}{\concproccst{(\concproccst{p_1}{p_2})}{p_3}}}
  \and
  \infern
  {\strut}
  {\jspeq{\concproccst{p}{\stopproccst}}{p}}
  \and
  \infern
  {\strut}
  {\jspeq{\concproccst{p_1}{p_2}}{\concproccst{p_2}{p_1}}}
\end{mathpar}

We allow new channels to be reordered, removed if unused, and hoisted to the top level provided they do not incur capture.
Rule E$_{12}$ is called \emph{scope extrusion}, since it allows the scope of $a$ to be extruded out of the concurrent composition.
\begin{mathpar}
  \infern
  {a_1 \ne a_2}
  {\jspeq{\tnewproccst{\tau_1}{a_1}{\tnewproccst{\tau_2}{a_2}{p}}}{\tnewproccst{\tau_2}{a_2}{\tnewproccst{\tau_1}{a_1}{p}}}}
  \and
  \infern
  {a \notin p}
  {\jspeq{\tnewproccst{\tau}{a}{p}}{p}}
  \and
  \infern
  {a \notin p_2}
  {\jspeq{\concproccst{(\tnewproccst{\tau}{a}{p_1})}{p_2}}{\tnewproccst{\tau}{a}{(\concproccst{p_1}{p_2})}}}
  \and
  \infern
  {a_1 \ne a_2}
  {\jspeq{\recvproccst{a_2}{x}{\tnewproccst{\tau}{a_1}{p}}}
    {\tnewproccst{\tau}{a_1}{\recvproccst{a_2}{x}{p}}}}
  \and
  \infern
  {\strut}
  {\jspeq{\indtycst{x}{P}}{\Subst{\indtycst{x}{P}}{x}{P}}}
  \and
  \infern
  {x' \not \in P}
  {\jspeq{\indtycst{x}{P}}{\indtycst{x'}{\Subst{x'}{x}{P}}}}
  \and
  \infern
  {a' \not \in P}
  {\jspeq{\tnewproccst{\tau}{a}{P}}{\tnewproccst{\tau}{a'}{\Subst{a'}{a}{P}}}}
\end{mathpar}

\subsubsection{Actions}

Actions also admit a statics, given by the judgment $\actionabt{\alpha}{\Sigma}$.

\setcounter{infercount}{1}
\renewcommand{\infern}[2]{\inferrule{#1}{#2}(\text{A}_{\arabic{infercount}}\stepcounter{infercount})}
\fbox{$\actionabt{\alpha}{\Sigma}$}
\begin{mathpar}
  \infern{ }{\actionabt{\silactcst}{\Sigma}}
  \and
  \infern{
    \jse{}{\Sigma, \Has{a}{\tau}}{e}{\tau}
  }{\actionabt{\tsndactcst{}{a}{e}}{\Sigma, \Has{a}{\tau}}}
  \and
  \infern{
    \jse{}{\Sigma, \Has{a}{\tau}}{e}{\tau}
  }{\actionabt{\trcvactcst{}{a}{e}}{\Sigma, \Has{a}{\tau}}}

\end{mathpar}

