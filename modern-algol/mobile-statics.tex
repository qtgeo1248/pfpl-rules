\subsection{Statics}

\subsubsection{Expressions}
Same as before

\subsubsection{Commands}

On the command level, the declaration command now demands the return type and the type of
the assignable to be \emph{mobile}. Intuitively, a type is mobile if its values does not depend on assignables. This
is to prevent the declared assignable to leave the scope of declaration and break safety.

\boxed{\Gamma \entails[\Sigma]{\IsOfM{m}{\tau}}}
\begin{mathpar}
\Infer*[\RCmdOf{\kw{dcl}}]
  {\Gamma \entails[\Sigma]{\IsOf{e}{\tau}} \\
   \Gamma \entails[\Sigma, \HasSym{a}{\tau}]{\IsOfM{m}{\tau'}} \\
   \IsMobile{\tau} \\
   \IsMobile{\tau'}}
  {\Gamma \entails[\Sigma]{\IsOfM{\tnewvarabt{\tau}{e}{a}{m}}{\tau'}}}
\end{mathpar}

The mobility judgment is defined inductively on the structure of $\tau$. Notably, command type is not mobile as it
can contain encapsulated accesses to the assignable. In addition, function types are not mobile, as the expression
in the body of the function may contain encapsulated commands.

\boxed{\IsMobile{\tau}}
\begin{mathpar}
\Infer*[$M_\nattycst$]
  {\strut}
  {\IsMobile{\nattycst}}

\Infer*[$M_\unittycst$]
  {\strut}
  {\IsMobile{\unittycst}}

\Infer*[$M_\voidtycst$]
  {\strut}
  {\IsMobile{\voidtycst}}\\

\Infer*[$M_{+}$]
  {\IsMobile{\tau_1} \\ \IsMobile{\tau_2}}
  {\IsMobile{\sumtycst{\tau_1}{\tau_2}}}

\Infer*[$M_{\times}$]
  {\IsMobile{\tau_1} \\ \IsMobile{\tau_2}}
  {\IsMobile{\prodtycst{\tau_1}{\tau_2}}}
\end{mathpar}
